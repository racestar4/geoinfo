\section{Installationsanleitung}
Dies ist eine Installationsanleitung für Ubuntu 18.04.
\subsection{Git Verzeichnis clonen oder ZIP Download}
Das Git Projekt clonen oder als Zip-Downloaden.
Über Github kann das Projekt als ZIP oder über Git clone heruntergeladen werden.

\begin{lstlisting}[frame=single,basicstyle=\small]
git clone https://github.com/racestar4/geoinfo.git

ZIP-Download: https://github.com/racestar4/geoinfo/archive/master.zip
\end{lstlisting}
Im Repository ist der geoserver und die geoinfoweb Verzeichnise. 

\subsection{Pfade setzen}
Den JAVAHOME Pfad zu der installieren Java Umgebung setzen.
Den GEOSERVERHOME Pfad zu dem beigefügtem Geoserver setzen.

Als Ubuntu-Beispiel (Pfade können abweichen) :
\begin{lstlisting}[frame=single,basicstyle=\small]
echo "export GEOSERVER_HOME=entpackterPfad/geoinfo/geoserver" >>/etc/environment
echo "JAVA_HOME=/usr/lib/jvm/java-11-openjdk-amd64" >>/etc/environment
\end{lstlisting}


\subsection{Start GeoServer und Frontend}
Sind die beiden Home Pfade gesetzt kann man in den ../geoserver/bin Ordner wechseln und die startup.sh ausführen.


Nun sollte der Geoserver auf localhost:8080 erreichbar sein.
Der Login kann über den Standard User : admin
PW: geoserver erfolgen.


Um das Frontend zum laufen zu bekommen muss man in den "geoinfoweb" Ordner des Repository wechseln.
 NPM muss für die folgenden Schritte installiert sein
Dann führt man die folgenden Befehle aus.
\begin{lstlisting}[frame=single,basicstyle=\small]
1. npm init
2. npm start
\end{lstlisting}
 "Npm start" kompiliert das Frontend und deployed es auf "http://localhost:1234/"
 
